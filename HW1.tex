\documentclass[10pt]{article}
 \usepackage[margin=0.5in]{geometry} 
\usepackage{amsmath,amsthm,amssymb,amsfonts}
 
\newcommand{\N}{\mathbb{N}}
\newcommand{\Z}{\mathbb{Z}}
 
\newenvironment{problem}[2][Problem]{\begin{trivlist}
\item[\hskip \labelsep {\bfseries #1}\hskip \labelsep {\bfseries #2.}]}{\end{trivlist}}
%If you want to title your bold things something different just make another thing exactly like this but replace "problem" with the name of the thing you want, like theorem or lemma or whatever
 
\begin{document}
 

 
Matt Fletcher
\\Homework TODO
\\Dr Siroj Kansakar
\\MA244-03

 
\begin{problem}{1.1.28}
$
\begin{array}
	 {rcr} x-y & = & 5 \\ 
	     3y+z & = & 11 \\
       	4z  & = & 8 \\
\end{array}
$
\end{problem}
 
\begin{proof}
Start with equation 3. Divide both sides by 4: \\
\[4z=8\]

\[\boxed{z=2}\]

Substitute this into the second equation:

\[3y+z = 11\]

\[3y+2=11\]

\[3y=9\]

\[\boxed{y=3}\]

Substitute this into the first equation:

\[x-y=5\]

\[x-3=5\]

\[\boxed{x=2}\]

\[\boxed{(x,y,z)=(2,3,2)}\]
\end{proof}



\begin{problem}{1.1.38}
	$
	\begin{array}
	{lcr} 3x+2y & = & 2 \\ 
	6x+4y & = & 14 \\
	\end{array}
	$
\end{problem}

\begin{proof}
	By inspection, the first equation appears to be very similar to the second. 
	
	Multiply the first equation by 2:
	
	\[6x+4y = 4\]
	
	However, the second equation says that $6x+4y = 14$. This set of equations is $\boxed{\text{inconsistent}}$.
\end{proof}

\begin{problem}{1.1.48}
$	
	\begin{array}
	{lcr} 
	x+y+z& = & 2 \\ 
	-x+3y+2z & = & 8 \\
	4x+y & = & 4 \\
	\end{array}
$
\end{problem}

\begin{proof}
Add 4 times the first equation and the negative of the third equation. 

\[3y+4z = 4\]

Next, add the first and second equations:
\[4y+3z = 10\]

This set of equations is easier to solve. 
	\[
	\begin{array}
	{lcr} 3y+4z & = & 4 \\ 
	4y+3z & = & 10 \\
	\end{array}
	\]

Add 4 times the first equation and $-3$ times the second. 

	\[
	\begin{array}
	{lcr} 12y+16z & = & 16 \\ 
	-12y-9z & = & -30 \\
	\end{array}
	\]
Adding these, $7z=-14$ and $\boxed{z = -2}$. Substitute this back into equation 2 (the unomodified one):

\[3y+(-8) = 4\]
\[\boxed{y = 4}\]

Substitute both of these into the original first equation:
\[x+2+-4 = 2\]
\[\boxed{x = 0}\]

Therefore, the final solution set is $\boxed{(x,y,z) = (0,4,-2)}$

\end{proof}


\begin{problem}{1.2.14}
$\begin{bmatrix}

1 & 2 & 1  &  0 \\
0 & 0 & 1 & -1 \\
0 & 0 & 0 & 0
\end{bmatrix}
$
\end{problem}

\begin{proof}
This matrix represents the following set of simultaneous equations:

	$	
	\begin{array}
	{lcr} 
	x+2y+z& = & 0 \\ 
	z & = & -1 \\
	0 & = & 0 \\
	\end{array}
	$
	
Substituting the value for $z$ in the second equation into the first equation, we get a linear equation in terms of $x$ and $y$. 

\[\boxed{x+2y = 1}\]

What this solution represents is a line in the $xy$-plane where every point on this line is an acceptable solution to the original matrix.  

\end{proof}

\begin{problem}{1.2.31}
	$	
	\begin{array}
	{lcr} 
	x  -3z & = & -2 \\ 
	3x+y-2z & = & 5 \\
	2x+2y+z & = & 4 \\
	\end{array}
	$
\end{problem}

\begin{proof}
Start by writing this as an augmented matrix:
$\begin{bmatrix}

1 & 0 & -3  &  -2 \\
3 & 1 & -2 & 5 \\
2 & 2 & 1 & 4
\end{bmatrix}
$

First, $R_3 = R_3 + R_1$


$\begin{bmatrix}

1 & 0 & -3  &  -2 \\
3 & 1 & -2 & 5 \\
3 & 2 & -2 & 2
\end{bmatrix}
$
\\
\\
Next, $R_3 = R_3 - R_2$

$\begin{bmatrix}

1 & 0 & -3  &  -2 \\
3 & 1 & -2 & 5 \\
3 & 2 & -2 & 2
\end{bmatrix}
$


	
	
	
	
	
	
	
	
	
	
	
	
	
	
%%ORIGINAL WORK, WRONG APPROACH	
\iffalse	
Add $-2$ times the second equation and the third equation:
\[-4x -3z = -6\]

Multiply through by $-1$. 
\[4x + 3z = 6\]

The original first equation and this equation are both in terms of $x$ and $z$.
	\[
	\begin{array}
	{lcr} 
	x-3z & = & -2 \\ 
	4x + 3z &=& 6 \\
	\end{array}
	\]
	
	Adding these two equations:
	\[5x = 4\]
\fi
\end{proof}

\end{document}




%StackOverflow links

%Making augmented matrix
% https://tex.stackexchange.com/questions/2233/whats-the-best-way-make-an-augmented-coefficient-matrix

%Making simulataneous equations
