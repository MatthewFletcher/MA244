\documentclass[10pt]{article}
 \usepackage[margin=0.5in]{geometry} 
\usepackage{amsmath,amsthm,amssymb,amsfonts}
 
\newcommand{\N}{\mathbb{N}}
\newcommand{\Z}{\mathbb{Z}}
 
\newenvironment{problem}[2][Problem]{\begin{trivlist}
\item[\hskip \labelsep {\bfseries #1}\hskip \labelsep {\bfseries #2.}]}{\end{trivlist}}
%If you want to title your bold things something different just make another thing exactly like this but replace "problem" with the name of the thing you want, like theorem or lemma or whatever
 
\begin{document}
 

 
Matt Fletcher
\\Homework TODO
\\Dr Siroj Kansakar
\\MA244-03

 
\begin{problem}{28}
$
\begin{array}
	 {rcr} x-y & = & 5 \\ 
	     3y+z & = & 11 \\
       	4z  & = & 8 \\
\end{array}
$
\end{problem}
 
\begin{proof}
Start with equation 3. Divide both sides by 4: \\
\[4z=8\]

\[\boxed{z=2}\]

Substitute this into the second equation:

\[3y+z = 11\]

\[3y+2=11\]

\[3y=9\]

\[\boxed{y=3}\]

Substitute this into the first equation:

\[x-y=5\]

\[x-3=5\]

\[\boxed{x=2}\]

\[\boxed{(x,y,z)=(2,3,2)}\]
\end{proof}



\begin{problem}{38}
	$
	\begin{array}
	{rcr} 3x+2y & = & 2 \\ 
	6x+4y & = & 14 \\
	\end{array}
	$
\end{problem}

\begin{proof}
	By inspection, the first equation appears to be very similar to the second. 
	
	Multiply the first equation by 2:
	
	\[6x+4y = 4\]
	
	However, the second equation says that $6x+4y = 14$. This set of equations is $\boxed{\text{inconsistent}}$.
\end{proof}


\begin{problem}{48TODO}
	$
	\begin{array}
	{rcr} 3x+2y & = & 2 \\ 
	6x+4y & = & 14 \\
	\end{array}
	$
\end{problem}

\begin{proof}
	By inspection, the first equation appears to be very similar to the second. 
	
	Multiply the first equation by 2:
	
	\[6x+4y = 4\]
	
	However, the second equation says that $6x+4y = 14$. This set of equations is $\boxed{\text{inconsistent}}$.
\end{proof}


\end{document}