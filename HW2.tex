\documentclass[8pt]{article}
 \usepackage[margin=0.5in]{geometry} 
\usepackage{amsmath,amsthm,amssymb,amsfonts}
 
\newcommand{\N}{\mathbb{N}}
\newcommand{\Z}{\mathbb{Z}}
 
\newenvironment{problem}[2][Problem]{\begin{trivlist}
\item[\hskip \labelsep {\bfseries #1}\hskip \labelsep {\bfseries #2.}]}{\end{trivlist}}
%If you want to title your bold things something different just make another thing exactly like this but replace "problem" with the name of the thing you want, like theorem or lemma or whatever
 
\begin{document}
 

 
\noindent Matt Fletcher
\\Homework 2
\\Dr Siroj Kansakar
\\MA244-03


%% Matrix formula 
%\[  \left| 
%\begin{array}{ccc}
%a & b & -c \\
%d & e & f \\
%g & h & i \end{array} 
%\right|\] 
%

 
\begin{problem}{12}
	Find the inverse of this matrix, if it exists. 
\[  \left| 
\begin{array}{ccc}
-1 & 1 \\
3 & -3  \\
\end{array} 
\right|\] 
\end{problem}
 
\begin{proof}
Find the determinant of the matrix. If the determinant is 0, then there is no inverse.
\[  det \left| 
\begin{array}{ccc}
-1 & 1 \\
3 & -3  \\
\end{array} 
\right| = (-1 \cdot -3) - (1 \cdot 3) = 3 - 3 = 0\]

\[\boxed{\text{This matrix has no inverse. }} \]


\end{proof}



\begin{problem}{14}
	Find the inverse of this matrix. 
\[  \left| 
\begin{array}{ccc}
1 & 2 & 2 \\
3 & 7 & 9 \\
-1 & -4 & 7 \end{array} 
\right|\] 

\end{problem}

\begin{proof}
Begin by finding the determinant. Call this matrix $A$. 

\[det(A) =1 \cdot \left| \begin{array}{ccc}7 & 9 \\-4 & -7  \\\end{array}\right| + 2\cdot \left|\begin{array}{ccc}3 & 9 \\-1 & -7  \\\end{array}\right| - 2 \cdot \left| \begin{array}{ccc}3 & 7 \\-1 & -4  \\\end{array} \right|\]

\[det(A) = 1(7\cdot -7 - 9\cdot -4) + 2(3\cdot -7 - 9\cdot -1) - 2(3\cdot -4 - 7\cdot -1) = 1.5\]

As the determinant is not $0$, there is an inverse to this matrix. 


\[\left[\begin{array}{ccc|ccc}  
1 & 2 & 2 & 1 & 0 & 0 \\  
3 & 7 & 9 & 0 & 1 & 0  \\
-1 & -4 & -7 & 0 & 0 & 1  
\end{array} \right]\]

$R_3 = R_3 + R_1$

\[\left[\begin{array}{ccc|ccc}  
1 & 2 & 2 & 1 & 0 & 0 \\  
3 & 7 & 9 & 0 & 1 & 0  \\
0 & -2 & -5 & 1 & 0 & 1  
\end{array} \right]\]

$R_2 = R_2 -3R_1$
\[\left[\begin{array}{ccc|ccc}  
1 & 2 & 2 & 1 & 0 & 0 \\  
0 & 1 & 3 & -3 & 1 & 0  \\
0 & -2 & -5 & 1 & 0 & 1  
\end{array} \right]\]

$R_3 = R_3 + 2R_2$

\[\left[\begin{array}{ccc|ccc}  
1 & 2 & 2 & 1 & 0 & 0 \\  
0 & 1 & 3 & -3 & 1 & 0  \\
0 & 0 & 1 & -5 & 2 & 1  
\end{array} \right]\]


$R_2 = R_2 - 3R_3$

\[\left[\begin{array}{ccc|ccc}  
1 & 2 & 2 & 1 & 0 & 0 \\  
0 & 1 & 0 & -4 & -5 & -3  \\
0 & 0 & 1 & -5 & 2 & 1  
\end{array} \right]\]

$R_1 = R_1 - 2R_2$
\[\left[\begin{array}{ccc|ccc}  
1 & 0 & 2 & 9 & 10 & 6 \\  
0 & 1 & 0 & -4 & -5 & -3  \\
0 & 0 & 1 & -5 & 2 & 1  
\end{array} \right]\]

\[R_1 = R_1 - 2R_3\]

\[\left[\begin{array}{ccc|ccc}  
1 & 0 & 0 & 19 & 6 & 4 \\  
0 & 1 & 0 & -4 & -5 & -3  \\
0 & 0 & 1 & -5 & 2 & 1  
\end{array} \right]\]


The right side of the augmented matrix is the inverse of the original matrix. 

\[ \boxed{ \left| 
\begin{array}{ccc}
19 & 6 & 4 \\
-4 & -5 & -3 \\
-5 & 2 & 1 \end{array}\right| }
\] 


\end{proof}


\begin{problem}{34}
	Find the inverse of this matrix. 
\[  \left| 
\begin{array}{ccc}
-12 & 3 \\
5 & -2  \\
\end{array} 
\right|\] 
\end{problem}

\begin{proof}
	Using the equation on page 66: \[A^{-1} = \frac{1}{ad-bc} \cdot \left| 
	\begin{array}{ccc}
	d & -b \\
	-c & a  \\
	\end{array} 
	\right|\]
	
	
	\[A^{-1} = \frac{1}{(-12 \cdot -2) - (3\cdot 5)}\left| 
	\begin{array}{ccc}
	-2 & -3 \\
	-5 & -12  \\
	\end{array} 
	\right|\]
	
	\[\frac{1}{9} \cdot \left| 
	\begin{array}{ccc}
	-2 & -3 \\
	-5 & -12  \\
	\end{array} 
	\right| \] 
	
	\[ \left| 
	\begin{array}{ccc}
	\frac{-2}{9} & \frac{1}{3} \\
	
	\frac{-5}{9} & \frac{-4}{3}  \\
	\end{array} 
	\right| \]
\end{proof}


\end{document}